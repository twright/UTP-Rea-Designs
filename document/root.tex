\documentclass[11pt,a4paper]{article}
\usepackage{isabelle,isabellesym}
\usepackage{fullpage}
\usepackage[usenames,dvipsnames]{color}
\usepackage{document}

% further packages required for unusual symbols (see also
% isabellesym.sty), use only when needed

\usepackage{amssymb}
  %for \<leadsto>, \<box>, \<diamond>, \<sqsupset>, \<mho>, \<Join>,
  %\<lhd>, \<lesssim>, \<greatersim>, \<lessapprox>, \<greaterapprox>,
  %\<triangleq>, \<yen>, \<lozenge>

\usepackage[english]{babel}
  %option greek for \<euro>
  %option english (default language) for \<guillemotleft>, \<guillemotright>

\usepackage{stmaryrd}
  %for \<Sqinter>

\usepackage{eufrak}
  %for \<AA> ... \<ZZ>, \<aa> ... \<zz> (also included in amssymb)

%\usepackage{textcomp}
  %for \<onequarter>, \<onehalf>, \<threequarters>, \<degree>, \<cent>,
  %\<currency>

% this should be the last package used
\usepackage{pdfsetup}

% urls in roman style, theory text in math-similar italics
\urlstyle{rm}
\isabellestyle{it}

% for uniform font size
%\renewcommand{\isastyle}{\isastyleminor}

\newcommand{\isactrlU}{\textit{\textbf{U}}}

\begin{document}

\title{Reactive Designs in Isabelle/UTP}

\author{Simon Foster \and James Baxter \and Ana Cavalcanti \and Jim Woodcock \and Samuel Canham}

\maketitle

\begin{abstract}
  Reactive designs combine the UTP theories of reactive processes and designs to characterise
  reactive programs. Whereas sequential imperative programs are expected to run until termination,
  reactive programs pause at instances to allow interaction with the environment using abstract
  events, and often do not terminate at all. Thus, whereas a design describes the precondition and 
  postcondition for a program, to characterise initial and final states, a reactive design also has 
  a ``pericondition'', which characterises intermediate quiescent observations. This gives rise
  to a notion of ``reactive contract'', which specifies the assumptions a program makes of its
  environment, and the guarantees it will make of its own behaviour in both intermediate and final
  observations. This Isabelle/UTP document mechanises the UTP theory of reactive designs, including
  its healthiness conditions, signature, and a large library of algebraic laws of reactive programming.
\end{abstract}

\tableofcontents

% sane default for proof documents
\parindent 0pt\parskip 0.5ex

\section{Introduction}

This document contains a mechanisation in Isabelle/UTP~\cite{Foster16a} of our theory of reactive designs. Reactive
designs form an important semantic foundation for reactive modelling languages such as Circus~\cite{Oliveira2005-PHD}.
For more details of this work, please see our recent paper~\cite{Foster17c}.

% generated text of all theories
\input{session}

% optional bibliography
\bibliographystyle{abbrv}
\bibliography{root}

\end{document}
